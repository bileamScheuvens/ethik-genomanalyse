\documentclass[12pt,a4paper]{article}
\usepackage[a4paper,top=20mm,bottom=25mm,left=20mm,right=40mm]{geometry}

\usepackage{wrapfig}
\usepackage{caption}
\usepackage{pdfpages}

\tolerance=1
\emergencystretch=\maxdimen
\hyphenpenalty=10000
\hbadness=10000

\graphicspath{ {./res/} }


\newenvironment{somebox}{\par\noindent%
   \begin{minipage}{\linewidth}}
   {\end{minipage}\par\vfill}


\usepackage[onehalfspacing]{setspace}

\begin{document}

\pagenumbering{gobble}
\begin{titlepage}
	\begin{center}
Hochschule Karlsruhe für Technik und Wirtschaft

\vspace{2cm}
\LARGE
\textbf{Ethikbumms}

\vspace{0.5cm}
\Large
Mit bisschen Datenschutz bumms

	\end{center}
\vfill
\large

Jenny Hilgenberg \& Bileam Scheuvens


\vspace{0.5cm}
Betreuer: Neumann \& Stengel


\end{titlepage}
	\newpage

\tableofcontents
	\newpage

\pagenumbering{arabic}
\setcounter{page}{3}
\section{Introduction}

	\newpage
\section{History}

\subsection{Disovery of Genes}
While initial speculation about genetics dates back to Hippocrates and Aristotle, who developed theories to explain inheritance\cite{mayr}, it was not until 1856 when Gregor Mendel experimented with cross breeding numerous pea plants to discover patterns in attribute inheritance\cite{mendel}, not knowing his actions would plant a seed within biology that would soon bloom to disrupt areas of daily life far beyond the bounds of science.
The groundbreaking discovery he made was that the phenotype of a given plant is not a mere blend of the visible attributes its parents held, but rather a seemingly random combination of the genetic lineage, meaning even attributes last observed several generations ago could be present\cite{mendel}.
Mendels work lead to a deeper understanding of the underlying genotype, dominant and recessive genes and was fundamental to the development of the techniques discussed in this paper.

\subsection{Genetic Engineering}
Human interference in evolution to bend nature to our will goes back at least to 2000 B.C. with strong evidence that the Assyrians artifically pollinated date trees\cite[p.633]{mayr}.
Since then several methods with the shared goal of reinforcing desirable characteristics and eradicating undesirables ones developed, most dominantly selective breeding, with the recent modification called mutation breeding, the practice of directly exposing organisms, usually seeds, to mutagenic matieral to positively augment the mutation rate and such the chance of new positive traits being present\cite{mutationbreeding}.
It can be argued that since selective and mutation breeding only ever affect coming generations it can hardly be seen as genetic modification of a single organism and is better labeled as a seperate method of eugenics.
From that perspective the first milestone of genetic engineering was the Avery–MacLeod–McCarty experiment in 1944, which proved that DNA was responsible for transformation in virus strains. It built upon Griffith's experiment from 1928, which showed that potent but dead virus strains can still transfer their properties to usually harmless strains, via something called the "transforming principle".
Avery, MacLeod and McCarty proved that the relevant component for this to work was DNA and not like previously assumed specific proteins, since the phenomenon persist when the solution is treated with protein dissolving substance, but seizes to persist when treated with DNA dissolving agents.
The most direct and controlled form of genetic modifiction is inserting strings of DNA into existing cells to transfer properties from one organism to another, deactivating- or cutting them out to suppress traits.
The most prominent example of this to date is the CRISPR/Cas method, which accomplishes this semi-reliably via the use of bacteria, but currently still struggles with off target effects, i.e. unforeseen changes outside of the intended locality.
While originally discovered in 1987, its use for genetic editing was only demonstrated in 2011 and since then a tremendous amount of research went into improving effectivity, safety and specificity, as well as finding applications in numerous in agriculture and medicine.

\subsection{Genome Analysis}
Necessary for the manipulation of genetic material is first understanding DNA and its function, which this paper will focus primarily on.
This breaks down into two steps.
Firstly DNA Sequencing, which means simply determining the sequence of DNA base pairs without interpretation. It enables the diagnosis of point mutations and similarity measurement, as well as forming the base for further analysis.
Secondly Genome-wide association studies (GWAS), which aim to map genetic variants to observable traits.

\section{Medical Sector}




	\newpage


 \section{Conclusion}

	\newpage
\pagenumbering{gobble}
\begin{thebibliography}{99}

\bibitem{mayr} Ernst, Mayr
\textbf{The Growth of Biological Thought} (Belknap Press, 1985)

\bibitem{mendel} Gregor Mendel,
\textbf{Versuche über Pflanzenhybriden} (Read at the meetings of February 8th, and March 8th, 1865)

\bibitem{mutationbreeding} Joanna Jankowicz-Cieslak, Thomas H. Tai, Jochen Kumlehn, Bradley J. Till
\textbf{Biotechnologies for Plant Mutation Breeding} (Springer Cham, 2017)

\bibitem{crispr} Anjanabha Bhattacharya, Vilas Parkhi, Bharat Char,
\textbf{CRISPR/Cas Genome Editing} (Springer Cham, 2020)

\bibitem{avery_macleod_mccarty} Oswald Avery, Colin MacLeod, Maclyn McCarty,
\textbf{Induction of Transformation by a Desoxyribonucleic Acid Fraction isolated from Pneumococcus type III} (Journal of Experimental Medicine, 1944)

\bibitem{schreiber} Hans-Peter Schreiber,
\textbf{Gentechnik, Genomanalyse und Ethik} (\textit{Ethik Med \textbf{10}} 1998)

\bibitem{rawal} Leena Rawal,  \& Sher Ali, 
\textbf{Genome Analysis and Human Health} (Springer Singapore 2017)

\bibitem{kumar} Sachil Kumar, 
\textbf{Forensic DNA Typing: Principles, Applications and Advancements} (Springer Singapore 2020)

\end{thebibliography}


\end{document}