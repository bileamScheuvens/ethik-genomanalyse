\documentclass[12pt,a4paper]{article}
\usepackage[a4paper,top=20mm,bottom=25mm,left=20mm,right=40mm]{geometry}

\usepackage{wrapfig}
\usepackage{caption}
\usepackage{pdfpages}

\tolerance=1
\emergencystretch=\maxdimen
\hyphenpenalty=10000
\hbadness=10000

\graphicspath{ {./res/} }


\newenvironment{somebox}{\par\noindent%
   \begin{minipage}{\linewidth}}
   {\end{minipage}\par\vfill}


\usepackage[onehalfspacing]{setspace}

\begin{document}

\pagenumbering{gobble}
\begin{titlepage}
	\begin{center}
Hochschule Karlsruhe für Technik und Wirtschaft

\vspace{2cm}
\LARGE
\textbf{Ethikbumms}

\vspace{0.5cm}
\Large
Mit bisschen Datenschutz bumms

	\end{center}
\vfill
\large

Jenny Hilgenberg \& Bileam Scheuvens


\vspace{0.5cm}
Betreuer: Neumann \& Stengel


\end{titlepage}
	\newpage

\tableofcontents
	\newpage

\pagenumbering{arabic}
\setcounter{page}{3}

	\newpage
\section{Historical and Technical Introduction}

\subsection{Discovery of Genes}
While initial speculation about genetics dates back to Hippocrates and Aristotle, who developed theories to explain inheritance\cite{mayr}, it was not until 1856 when Gregor Mendel experimented with cross breeding numerous pea plants to discover patterns in attribute inheritance\cite{mendel}, not knowing his actions would plant a seed within biology that would soon bloom to disrupt areas of daily life far beyond the bounds of science.
The groundbreaking discovery he made was that the phenotype of a given plant is not a mere blend of the visible attributes its parents held, but rather a seemingly random combination of the genetic lineage, meaning even attributes last observed several generations ago could be present\cite{mendel}.
Mendels work lead to a deeper understanding of the underlying genotype, dominant and recessive genes and was fundamental to the development of the techniques discussed in this paper.

\subsection{Genetic Engineering}
Human interference in evolution to bend nature to our will goes back at least to 2000 B.C. with strong evidence that the Assyrians artifically pollinated date trees\cite[p.633]{mayr}.
Since then several methods with the shared goal of reinforcing desirable characteristics and eradicating undesirables ones developed, most dominantly selective breeding, with the recent modification called mutation breeding, the practice of directly exposing organisms, usually seeds, to mutagenic matieral to positively augment the mutation rate and such the chance of new positive traits being present\cite{mutationbreeding}.
It can be argued that since selective and mutation breeding only ever affect coming generations it can hardly be seen as genetic modification of a single organism and is better labeled as a seperate method of eugenics.
From that perspective the first milestone of genetic engineering was the Avery–MacLeod–McCarty experiment in 1944, which proved that DNA was responsible for transformation in virus strains. It built upon Griffith's experiment from 1928, which showed that potent but dead virus strains can still transfer their properties to usually harmless strains, via something called the "transforming principle".
Avery, MacLeod and McCarty proved that the relevant component for this to work was DNA and not like previously assumed specific proteins, since the phenomenon persist when the solution is treated with protein dissolving substance, but seizes to persist when treated with DNA dissolving agents.
The most direct and controlled form of genetic modifiction is inserting strings of DNA into existing cells to transfer properties from one organism to another, deactivating- or cutting them out to suppress traits.
The most prominent example of this to date is the CRISPR/Cas method, which accomplishes this semi-reliably via the use of bacteria, but currently still struggles with off target effects, i.e. unforeseen changes outside of the intended locality.
While originally discovered in 1987, its use for genetic editing was only demonstrated in 2011 and since then a tremendous amount of research went into improving effectivity, safety and specificity, as well as finding applications in numerous in agriculture and medicine.

\subsection{Genome Analysis}
Necessary for the manipulation of genetic material is first understanding DNA, its function and effects of deviation from the norm, which this paper will focus primarily on.
Fundamental understanding breaks down into two steps.
Firstly DNA Sequencing, which means simply determining the sequence of DNA base pairs without interpretation. It enables the diagnosis of point mutations and similarity measurement, as well as forming the base for further analysis.
Secondly gene association studies, which aim to map genetic variants to observable traits and come in two varieties. 
Genome-wide association studies (GWAS) hope to uncover any correlation between strings of DNA and phenotype, while specific genetic association or gene linkage studies focus on specific variants of a gene (alleles) and examine whether they frequently cooccur with disease\cite{gwas}.


	\newpage

\section{Who Are You}

\subsection{DNA and Identity}

	\newpage

\section{What Have You Done}

\subsection{Forensics}
The controversy around privacy versus surveillance applies to genome analysis as much as to any sensitive technology.

	\newpage


\section{What Does Your Future Look Like}

\subsection{Medical Sector}

\subsection{Insurance}

	\newpage

\section{Consequences for Society: Social Compatibility}

\subsection{Eugenics}
Eugenics, subject to study under the field of bioethics, has been of concern to humanity since the beginning of civilization. While it literally translates to "good birth", it encompasses all attempts at improvement to the biological nature of the human race as a whole\cite{eugenics}. \newline
History has seen it as the justification for a number of crude methods like infanticide in Sparta and forced sterilization under the Nazi regime, some of which are still carried out around the world today. Yet the tools we are soon to wield at the current pace of development towards controlled genetic manipulation go far beyond that, making a new form of eugenics a viable option towards social good. \newline
In itself maximizing potential for generations to come, whether it be with improved intelligence, longer and healthier lives or stress tolerance seems like a noble goal, however one must not belittle possible consequences.

A well reflected perspective enforces the insight that our social, political and ethical ideologies at any given point in time are never fully refined and future generations might look down on views we currently proudly uphold. Irreversibly changing our collective genetics could lead us in a direction we might one day want to have never followed.

Eradicating hereditary disease and disability in a nonviolent way is a seemingly obvious step towards a world with less suffering.
Examining this from a moral standpoint one cannot overlook the strong parallels to the nonidentity problem, which describes the question whether preventing a flawed existence filled with suffering and to replace it with a less flawed and nonidentical one is desirable from a moral point of view\cite{nonidentity}.

Additionally there are numerous short term consequences. Preventing some but not all disease would result in amplification of marginalization, possibly increasing existing discrimination. \newline
The idea that disability is inherently wrong and needs to be fixed in itself is a questionable statement at best and strongly inhumane at worst\cite{schreiber}.

Assuming policies would be enacted on a national level instead of as a global initiative, practicing eugenics via genetic manipulation would deepen the divide between developed and developing nations causing additional inequality, alongside the social inequality unavoidibly caused by liberal eugenics, the approach of leaving the use of enhancements to choice.\newline
The opposite, obligatory genetic enhancement, would unearth even bigger concerns regarding bodily and reproductive autonomy\cite{eugenics}.

Artificially narrowing the human genepool might also have unforeseen sideeffects. As diversity is one of the driving agents of natural selection, careless intervention could inhibit our natural ability to defend against novel threats and disease\cite{mayr}.
It might quite literally be the end of evolution and the start of a new biological era. \newline
There are also outcries from a religious perspective arguing that taking responsibility away from god is divinely forbidden\cite{eugenics}.

All of these have to be weighed against the possible good that genetic enhancement might bring and making a definitive recommendation free of moral ambiguity is most likely impossible.





\subsection{Totalitarian Police State}

	\newpage


 \section{Conclusion}

	\newpage

\pagenumbering{gobble}
\begin{thebibliography}{99}

\bibitem{mayr} Ernst, Mayr
\textbf{The Growth of Biological Thought} (Belknap Press, 1985)

\bibitem{mendel} Gregor Mendel,
\textbf{Versuche über Pflanzenhybriden} (Read at the meetings of February 8th, and March 8th, 1865)

\bibitem{mutationbreeding} Joanna Jankowicz-Cieslak, Thomas H. Tai, Jochen Kumlehn, Bradley J. Till
\textbf{Biotechnologies for Plant Mutation Breeding} (Springer Cham, 2017)

\bibitem{crispr} Anjanabha Bhattacharya, Vilas Parkhi, Bharat Char,
\textbf{CRISPR/Cas Genome Editing} (Springer Cham, 2020)

\bibitem{avery_macleod_mccarty} Oswald Avery, Colin MacLeod, Maclyn McCarty,
\textbf{Induction of Transformation by a Desoxyribonucleic Acid Fraction isolated from Pneumococcus type III} (Journal of Experimental Medicine, 1944)

\bibitem{gwas} 
\textbf{GWA (genome-wide association, GWAS)} (Springer Dordrecht, 2008)

\bibitem{schreiber} Hans-Peter Schreiber,
\textbf{Gentechnik, Genomanalyse und Ethik} (\textit{Ethik Med \textbf{10}} 1998)

\bibitem{rawal} Leena Rawal,  \& Sher Ali, 
\textbf{Genome Analysis and Human Health} (Springer Singapore 2017)

\bibitem{kumar} Sachil Kumar, 
\textbf{Forensic DNA Typing: Principles, Applications and Advancements} (Springer Singapore 2020)

\bibitem{eugenics} Sara Goering,
\textbf{Eugenics} (Stanford Encyclopedia of Philosophy, 2014)

\bibitem{nonidentity} Melinda Roberts,
\textbf{The Nonidentity Problem} (Stanford Encyclopedia of Philosophy, 2009)

\end{thebibliography}


\end{document}