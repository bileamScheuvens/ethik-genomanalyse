\documentclass[conference]{IEEEtran}
\IEEEoverridecommandlockouts
% The preceding line is only needed to identify funding in the first footnote. If that is unneeded, please comment it out.
\usepackage{cite}
\usepackage{amsmath,amssymb,amsfonts}
\usepackage{algorithmic}
\usepackage{graphicx}
\usepackage{textcomp}
\usepackage{xcolor}

\usepackage{wrapfig}
\usepackage{caption}
\usepackage{pdfpages}


\graphicspath{ {./res/} }


\begin{document}
\pagenumbering{gobble}

\title{Weighing Chances and Use Against Risks and Misuse of Emerging Genome Analysis and Manipulation Technology}

\author{\IEEEauthorblockN{1\textsuperscript{st} Benedict Bileam Scheuvens}
\IEEEauthorblockA{\textit{Data Science} \\
\textit{Hochschule Karlsruhe für Technik und Wirtschaft}\\
Karlsruhe, Germany \\
scbe1047@h-ka.de}
\and
\IEEEauthorblockN{2\textsuperscript{nd} Jennifer Hilgenberg}
\IEEEauthorblockA{\textit{Data Science} \\
\textit{Hochschule Karlsruhe für Technik und Wirtschaft}\\
Karlsruhe, Germany \\
hije1012@h-ka.de}
}

\maketitle

\begin{abstract}
A few centuries ago, life love and a the universe were a mystery, but driven by deeply rooted human curiosity we are beginning to untangle the mess and understand the foundation that grew consciousness out of thin air. One piece of this puzzle is DesoxyriboNucleic Acid. The genetic material that after being refined by millenia of pure chance and survival holds the blueprint to every living creature on our planet. With growing understanding about its structure and the technological advancements that gave us tools to manipulate it, humanity has set its foot in the doorstep to a world in which DNA is no different than a LEGO brick and nothing is out of reach.
This paper aims to explore aspects of how this applied genetics will impact us in regard to ethics and data privacy. 
Since great responsibility comes with great power, Genetic engineering is as much blessing as curse and must be used with caution, while genome analysis unveils a tremendous amount of information not just about us, but about our relatives as well and enables more and more accurate predictions about the future.
\end{abstract}
\section{Historical and Technical Introduction}

\subsection{Discovery of Genes}
While initial speculation about genetics dates back to Hippocrates and Aristotle, who developed theories to explain inheritance\cite{mayr}, it was not until 1856 when Gregor Mendel experimented with cross breeding numerous pea plants to discover patterns in attribute inheritance\cite{mendel}, not knowing his actions would plant a seed within biology that would soon bloom to disrupt areas of daily life far beyond the bounds of academic science.
The groundbreaking discovery he made was that the activate traits (phenotype) of a given plant is not a mere blend of the visible attributes its parents held, but rather a seemingly random combination of the genetic lineage, meaning even attributes last observed several generations ago could be present\cite{mendel}.
Mendels work lead to a deeper understanding of the underlying mechanisms of inheritance, dominant and recessive genes (strings of DNA that are translated into a different product) and was fundamental to the development of the techniques discussed in this paper.

\subsection{Genetic Engineering}
Human interference in evolution to bend nature to our will goes back at least to 2000 B.C. with strong evidence that the Assyrians artifically pollinated date trees\cite[p.633]{mayr}.
Since then several methods with the shared goal of reinforcing desirable characteristics and eradicating undesirables ones developed, most dominantly selective breeding, with the recent modification called mutation breeding, the practice of directly exposing organisms, usually seeds, to mutagenic matieral to positively augment the mutation rate and such the chance of new positive traits being present\cite{mutationbreeding}.
It can be argued that since selective and mutation breeding only ever affect coming generations it can hardly be seen as genetic modification of a single organism and is better labeled as a seperate method of eugenics.
From that perspective the first milestone of genetic engineering was the Avery–MacLeod–McCarty experiment in 1944, which proved that DNA was responsible for transformation in virus strains. It built upon Griffith's experiment from 1928, which showed that potent but dead virus strains can still transfer their properties to usually harmless strains, via something called the "transforming principle".
Avery, MacLeod and McCarty proved that the relevant component for this to work was DNA and not like previously assumed specific proteins, since the phenomenon persist when the solution is treated with protein dissolving substance, but seizes to persist when treated with DNA dissolving agents.
The most direct and controlled form of genetic modifiction is inserting strings of DNA into existing cells to transfer properties from one organism to another, deactivating- or cutting them out to suppress traits.
The most prominent example of this to date is the CRISPR/Cas method, which accomplishes this semi-reliably via the use of bacteria, but currently still struggles with off target effects, i.e. unforeseen changes outside of the intended locality.
While originally discovered in 1987, its use for genetic editing was only demonstrated in 2011 and since then a tremendous amount of research went into improving effectivity, safety and specificity, as well as finding applications in numerous in agriculture and medicine.
It's important to keep in mind that current methods operate on single cells, which while passing the modified DNA on during mitosis, are clearly outnumbered by non modified cells\cite{crispr}.
This can be worked around in three ways.
Modify a sufficient number of cells (most likely infeasible for broad changes), modify in an early stage of development, where only a number of cells exist, or germline editing, which means editing the parents germ cells (cells involved in reproduction) in way that the offspring inherits the changes.
The third approach differs from the other (so called somatic / nonheritable) techniques in that it can irreversibly affect future generations of offspring with undoubtedly vast consequences, which is why there is widespread advocation to refrain from practicing it in humans\cite{dontedit}.

\subsection{Genome Analysis}
Necessary for the manipulation of genetic material is first understanding DNA, its function and effects of deviation from the norm, which this paper will focus primarily on.
Fundamental understanding breaks down into two steps.
Firstly DNA Sequencing, which means simply determining the sequence of DNA base pairs without interpretation. It enables the diagnosis of point mutations and similarity measurement, as well as forming the base for further analysis.

Secondly gene association studies, which aim to map genetic variants to observable traits and come in two varieties. 
Genome-wide association studies (GWAS) hope to uncover any correlation between strings of DNA and phenotype, while specific genetic association or gene linkage studies focus on specific variants of a gene (alleles) and examine whether they frequently cooccur with disease\cite{gwas}.
This endeavour presents numerous challenges stemming from the sheer complexity of the mechanisms at hand.
Part of the reason this differs from an exact science is changes an organisms environment has on its genome. It is estimated as much 98\% of human DNA is noncoding, id est lacks any apparent purpose\cite{noncoding}. Some of it however is not entirely disfunctional, rather it is dynamically activated and silenced in a process known as epigenetics changes by so called regulatory genes\cite{epigenetics}.
This means the exact same sequence of DNA can express itself differently, making direct conclusions about the significance of a specific snippet fall short of reliability.
In 2003 with the official completion of the Human Genome Project (HGP) this has been accomplished\cite{HGPFAQ}.
The human genome consists of approximately 3.3 billion basepairs, by cross referencing a healthy genome, single point mutations can be effortlessly identified in theory, especially with modern high-throughput analysis tools\cite{rawal}.
Unfortunately in the context of genetics this notion of a healthy genome is vague. Not only does a healthy genome not exist, simply averaging the genome of a few relatively healthy adults would do it no justice. Changes in DNA can also come in the form of deletion or addition of bases rendering traditional similarity metrics useless as an applicable method would have to be alignment free, in other words agnostic to the exact position in sequence, rather it should recognize subsequences or repetitions anywhere they show up and compare the frequency. Through the nature of mutation, the 'average genome' also varies widely regionally, the HGP in particular faced criticism for overrepresenting caucasian DNA, owed to the fact that it was conducted on anonymous donations of DNA in the USA\cite{HGPcritic}.
It should therefore be seen more like a reference frame to possibly detect deviation.
Cataloguing mutations is useful but by itself often provides little insight, as chances are the mutation occured a noncoding part.
Even when mutation occurs in regulatory sequences or as part of a gene single mutations these usually have little effect.
Its important to keep in mind that with mutations being so common, each cell has a slightly different sequence. The only mutations they share are those that occured in the germ cells that formed the zygote or in the very early stages afterwards.
Germline mutations occur roughly $\frac{number of base pairs}{2}*10^{-9}$ times per year. In humans that works out to about 1.65 inheritable mutations per year per parent, but when examining the genome of adult humans there is sampling bias at play. Single point mutations with drastic changes usually just render the cells nonviable, consequently an egg being fertilized and developing long enough to be given birth to indicates a mostly intact genome\cite{scally}.
Somatic mutations occur in the order of magnitute of $ 2*\frac{number of cell divisions}{10^8 base pairs}$, which with 30 trillion cells equates to roughly 500 trillion per year\cite{somaticmutation}.
These usually have no or very little effect, are repaired by the cells dna repair mechanisms or kill the cell either by inhibiting vital function or by causing cell suicide\cite{dnarepair}.
The unlikely unlikely case that DNA damage results in neither of these scenarios and the cell instead starts rapidly reproducing is what we call cancer.


\section{Who Are You}

\subsection{DNA and Identity}

\section{What Have You Done}

\subsection{Forensics}
The controversy around privacy versus surveillance applies to genome analysis as much as to any sensitive technology.



\section{What Does Your Future Look Like}

\subsection{Medical Sector}
As preventing disease is usually preferable over treating it, a substantial amount of medicine focuses on early detection and prevention. Especially for the class of mendelian disorders, which are characterized by being linkable to a single gene defect, genetic testing has proven to make diagnosis easy and reliable \cite{rawal}.
However while a grand number of such diseases exist, they are usually rare so testing might only pay off if the defect is known to run in the family \cite{publicopinion}.
Much bigger potential lies in preventing common disease with either a genetic component or whose causes are spread over several genes and are therefore  harder to pinpoint. This uncertainty neccessarily introduces stochastics and the method known as genetic risk assessment \cite{gwas}.
Advances here have sprouted the field of pharmagenomics, which utilizes genetic information to practice precision medicine.
Just like ethnicity, genetic profile has been shown to affect drug response, thus using this knowledge to personalize prescriptions can increase effectiveness and lower cost and side effects \cite{costsaving}.
Predictive genetic testing does not come without concerns from both experts and the public, the most prominent ones will be discussed in the following:

\subsubsection{Patient Autonomy}
The primary role of any new medical instrument should be to serve, therefore patient autonomy and informed consent are non negotiable, which neccessitates public education.
Broad, as well as targeted at populations most susceptible to specific genetic conditions.
Equally as important is access to testing facilities themselves as well as short wait times for both test and treatment\cite{publicopinion}.
\subsubsection{Genetic Discrimation}
As will be discussed in the following chapter, subpar treatment grounded in genome analysis results is no better than discrimination on the basis of sex, race, age or religion.
\subsubsection{Commercial Use}
Profit as the primary incentive manages to corrupt even the most honourable endeavour.
Commercial gene testing and therapy raises the same questions about motive and ethical obligation to help that plagued the medical ethics for millenia.
Testing costs being inflated because of patents can hardly be in accordance with public interest, which sparks discussion about the next point\cite{publicopinion}.
\subsubsection{Use of Public Resources}
If private patents are to be avoided, public resources have to be used towards research in the applicable area.
Distributing those in a fair way however proves difficult. Developing tests is expensive so some conditions need to be prioritized, which in turn requires weighing of different factors. The initial cost of development is just as important as benefit it brings. This is comprised of the reliability of the result, as well as therapy options versus their respective cost. An unreliable result is equally useless from a public health perspective as knowledge about a genetic death sentence.
Additionally as many tests would only ever affect a small number of people, some argue that the money should either be used for more general genetic tests or towards programs that target public health in a broader sense like adressing diet\cite{publicopinion}.
\subsubsection{Patient Responsibilities}
While a general right to data privacy in medicine makes sense for traditional diagnostics, one might argue that since DNA test results have immediate consequences for relatives theres at least in part a responsibility to inform said relatives of any risks they themselves might face, comparable to infectious disease or STDs\cite{publicopinion}.
This apparent obligation to warn stands in direct conflict with the testers pledge of silence and the testees right to privacy, a conflict that could in theory be resolved with institutions for anonymous testing and subsequent information of concerned parties analogous to HIV testing.
\subsubsection{Psychological Risks}
The psychological weight associated with carrying near certain information about the future is not to be underestimated in the context of genetic testing either.
Just as a cancer diagnosis, discovering to be at high risk of devoloping potentially lethal disease can be devastating, especially if treatment options are limited\cite{publicopinion}.
Depending on the evolution of public debate on the topic, a scenario in which gene therapy is frowned upon in a similar manner to termination of pregnancy by groups within society is also imaginable.
This would add additional stress and force people at risk for genetic disease to make a choice between social acceptability and their own health.


\subsection{Insurance}
Insurance as a business walks a fine line between solidarity and preying on the vulnerable. Risk and how we manage it as a society is studied under the name of insurance ethics, but to date the literature is sparse.
While public insurance in practice is a form of wealth redistribution, private insurance is in essence paying for peace of mind\cite{insuranceethics}.
However this social safety net only functions as long as there is uncertainty, in fact it functions better the more uncertainty there is.
This becomes obvious in consideration of the idea that clarity about the future replaces the insurance sector.
In a fully deterministic world, omniscient private insurance companies would never lose money on a customer, as premiums could be precisely the cost said customer will produce over their contract, plus a small markup; in other words, not providing any insurance.
Keeping this in mind it seems logical that it's in the best interest of society at large to keep insurance companies mostly in the dark.
At least in the case of health insurance, this darkness is threatened by introducing gene testing aided risk analysis.
The inclusion of factors like genetic predisposition to disease in risk analysis is an incredibly helpful tool to doctors when choosing appropriate treatment and taking precautions, but becomes detrimental when accessed by insurance companies\cite{insurancegenes}.
Legal action to protect against this threat is already being taken, with the USA introducing the Genetic Information Nondiscrimination Act in 2008\cite{insurancegenes}.
It is vital to human development that we retain trust in and refine the methodology of genetic testing, while at the same time making sure the sensitive informations is stored discretely and genetic discrimination is kept under control.

\section{Consequences for Society: Social Compatibility}

\subsection{Eugenics}
Eugenics, subject to study under the field of bioethics, has been of concern to humanity since the beginning of civilization. While it literally translates to "good birth", it encompasses all attempts at improvement to the biological nature of the human race as a whole\cite{eugenics}. \newline
History has seen it as the justification for a number of crude methods like infanticide in Sparta and forced sterilization under the Nazi regime, some of which are still carried out around the world today. Yet the tools we are soon to wield at the current pace of development towards controlled genetic manipulation go far beyond that, making a new form of eugenics a viable option towards social good. \newline
For the purpose of eugenics, genetic manipulation and genome analysis of fertilized cells to select which one to implement are almost equivalent, as both manipulate chance in our favor. The exact method is therefore irrelevant to the discussion in this chapter.
In itself maximizing potential for generations to come, whether it be with improved intelligence, longer and healthier lives or stress tolerance seems like a noble goal, however one must not belittle possible consequences.

A well reflected perspective enforces the insight that our social, political and ethical ideologies at any given point in time are never fully refined and future generations might look down on views we currently proudly uphold. Irreversibly changing our collective genetics could lead us in a direction we might one day want to have never followed.

Eradicating hereditary disease and disability in a nonviolent way is a seemingly obvious step towards a world with less suffering.
Examining this from a moral standpoint one cannot overlook the strong parallels to the nonidentity problem, which describes the question whether preventing a flawed existence filled with suffering and to replace it with a less flawed and nonidentical one is desirable from a moral point of view\cite{nonidentity}.

Additionally there are numerous short term consequences. Preventing some but not all disease would result in amplification of marginalization, possibly increasing existing discrimination. \newline
The idea that disability is inherently wrong and needs to be fixed in itself is a questionable statement at best and strongly inhumane at worst\cite{schreiber}.

Assuming policies would be enacted on a national level instead of as a global initiative, practicing eugenics via genetic manipulation would deepen the divide between developed and developing nations causing additional inequality, alongside the social inequality unavoidibly caused by liberal eugenics, the approach of leaving the use of enhancements to choice.\newline
The opposite, obligatory genetic enhancement, would unearth even bigger concerns regarding bodily and reproductive autonomy\cite{eugenics}.

Artificially narrowing the human genepool might also have unforeseen sideeffects. As diversity is one of the driving agents of natural selection, careless intervention could inhibit our natural ability to defend against novel threats and disease\cite{mayr}.
It might quite literally be the end of evolution and the start of a new biological era. \newline
There are also outcries from a religious perspective arguing that taking responsibility away from god is divinely forbidden\cite{eugenics}.

All of these have to be weighed against the possible good that genetic enhancement might bring and making a definitive recommendation free of moral ambiguity is most likely impossible.
It quickly becomes a discussion about what it is we as a society are trying to accomplish. For minimizing future suffering overall eugenics might be the obvious choice.
On the other hand a statement about maximizing joy cannot be made as easily. While healthier lives are certainly desirable, it would quickly become the status quo and no longer something to be grateful for.
With standards rising it could be equally difficult to fit in so theres no reason to believe a genetically enhanced society would constantly be ecstatic about its accomplishments.
Futhermore most of the prior arguments assume total confidence in the human ability to manipulate genes, which quite frankly is far from being warranted.
Theres good reason most of the biological world has agreed to put experimentation in humans on hold\cite{dontedit}.


 \section{Conclusion}

\pagenumbering{gobble}
\begin{thebibliography}{99}

\bibitem{mayr} Ernst, Mayr
\textbf{The Growth of Biological Thought} (Belknap Press, 1985)

\bibitem{mendel} Gregor Mendel,
\textbf{Versuche über Pflanzenhybriden} (Read at the meetings of February 8th, and March 8th, 1865)

\bibitem{mutationbreeding} Joanna Jankowicz-Cieslak, Thomas H. Tai, Jochen Kumlehn, Bradley J. Till
\textbf{Biotechnologies for Plant Mutation Breeding} (Springer Cham, 2017)

\bibitem{crispr} Anjanabha Bhattacharya, Vilas Parkhi, Bharat Char,
\textbf{CRISPR/Cas Genome Editing} (Springer Cham, 2020)

\bibitem{dontedit} E. Lanphier, F. Urnov, S. Haecker,
\textbf{Don’t edit the human germ line} (Nature, 2015)

\bibitem{avery_macleod_mccarty} Oswald Avery, Colin MacLeod, Maclyn McCarty,
\textbf{Induction of Transformation by a Desoxyribonucleic Acid Fraction isolated from Pneumococcus type III} (Journal of Experimental Medicine, 1944)

\bibitem{gwas} 
\textbf{GWA (genome-wide association, GWAS)} (Springer Dordrecht, 2008)

\bibitem{HGPFAQ}
\textbf{Human Genome Project FAQ} (https://www.genome.gov/human-genome-project/Completion-FAQ)

\bibitem{HGPcritic} Raymond E. Spier,
\textbf{The Human Genome Project Under the Microscope} (Science and Engineering Ethics, 1998)

\bibitem{scally} A. Scally,
\textbf{The mutation rate in human evolution and demographic inference} (Elsevier BV, 2016)

\bibitem{dnarepair} H. Lodish, A. Berk, P. Matsudaira, C.A. Kaiser, M. Krieger, M.P. Scott, S.L. Zipursky, J. Darnell 
\textbf{Molecular Biology of the Cell} (New York WH Freeman, 2004)

\bibitem{somaticmutation} Satoshi Oota,
\textbf{Somatic mutations – Evolution within the individual} (ScienceDirect, 2019)

\bibitem{noncoding} F. Costa,
\textbf{Non-coding RNAs, Epigenomics, and Complexity in Human Cells} (Caister Academic Press, 2012)

\bibitem{epigenetics} Alexander Meissner, Jörn Walter,
\textbf{Epigentic Mechanism in Cellular Reprogramming} (Springer Berlin-Heidelberg, 2015)

\bibitem{schreiber} Hans-Peter Schreiber,
\textbf{Gentechnik, Genomanalyse und Ethik} (\textit{Ethik Med \textbf{10}}, 1998)

\bibitem{rawal} Leena Rawal,  \& Sher Ali, 
\textbf{Genome Analysis and Human Health} (Springer Singapore, 2017)

\bibitem{kumar} Sachil Kumar, 
\textbf{Forensic DNA Typing: Principles, Applications and Advancements} (Springer Singapore, 2020)

\bibitem{publicopinion} Douglas Martin, Heather Greenwood, Jeff Nisker,
\textbf{Public Perception of Ethical Issues Regarding Adult Predictive Genetic Testing} (Health Care Anal, 2010)

\bibitem{costsaving} 
\textbf{Genetic testing - a powerful cost saving tool} (Pharmacoecon, 2013)

\bibitem{insuranceethics} Aaron Doyle,
\textbf{Introduction: Insurance and Business Ethics} (Carleton University, 2012)

\bibitem{insurancegenes} Michelle Lane, Ida Ngueng Feze \& Yann Joly,
\textbf{Genetics and Personal Insurance: the Perspectives of Canadian Cancer Genetic Counselors} (Journal of Genetic Counseling, 2015)

\bibitem{eugenics} Sara Goering,
\textbf{Eugenics} (Stanford Encyclopedia of Philosophy, 2014)

\bibitem{nonidentity} Melinda Roberts,
\textbf{The Nonidentity Problem} (Stanford Encyclopedia of Philosophy, 2009)

\end{thebibliography}


\end{document}