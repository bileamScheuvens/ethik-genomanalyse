\documentclass[12pt,a4paper]{article}
\usepackage[a4paper,top=20mm,bottom=25mm,left=20mm,right=40mm]{geometry}

\usepackage{wrapfig}
\usepackage{caption}
\usepackage{pdfpages}

\tolerance=1
\emergencystretch=\maxdimen
\hyphenpenalty=10000
\hbadness=10000

\graphicspath{ {./res/} }


\newenvironment{somebox}{\par\noindent%
   \begin{minipage}{\linewidth}}
   {\end{minipage}\par\vfill}


\usepackage[onehalfspacing]{setspace}

\begin{document}

\pagenumbering{gobble}
\begin{titlepage}
	\begin{center}
Hochschule Karlsruhe für Technik und Wirtschaft

\vspace{2cm}
\LARGE
\textbf{Ethikbumms}

\vspace{0.5cm}
\Large
Mit bisschen Datenschutz bumms

	\end{center}
\vfill
\large

Jenny Hilgenberg \& Bileam Scheuvens


\vspace{0.5cm}
Betreuer: Neumann \& Stengel


\end{titlepage}
	\newpage

\tableofcontents
	\newpage

\pagenumbering{arabic}
\setcounter{page}{3}
\section{Introduction}

	\newpage
\section{History}

While initial speculation about genetics dates back to Hippocrates and Aristotle, who developed theories to explain inheritance\cite{mayr}, it was not until 1856 when Gregor Mendel experimented with cross breeding numerous pea plants to discover patterns in attribute inheritance\cite{mendel}, not knowing his actions would plant a seed within biology that would soon bloom to disrupt areas of daily life far beyond the bounds of science.
	\newpage


 \section{Conclusion}

	\newpage
\pagenumbering{gobble}
\begin{thebibliography}{99}

\bibitem{mayr} Mayr, Ernst,
\textbf{The Growth of Biological Thought} (Belknap Press, 1985)

\bibitem{mendel} Mendel, Gregor
\textbf{Experiments in Plant Hybridization} (Read at the meetings of February 8th, and March 8th, 1865)

\bibitem{schreiber} Schreiber, Hans-Peter, 
\textbf{Gentechnik, Genomanalyse und Ethik} (\textit{Ethik Med \textbf{10}} 1998)

\bibitem{rawal} Rawal, Leena \& Ali, Sher,
\textbf{Genome Analysis and Human Health} (Springer Singapore 2017)

\end{thebibliography}


\end{document}